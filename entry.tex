
% !TeX program = pdflatex
\documentclass[12pt,a4paper,oneside]{scrreprt}

% ---------- Packages ----------
\usepackage[T1]{fontenc}
\usepackage[utf8]{inputenc}
\usepackage[ngerman]{babel}
\usepackage{lmodern}
\usepackage{geometry}
\usepackage{setspace}
\usepackage{hyperref}
\usepackage{graphicx}
\usepackage{caption}
\usepackage{booktabs}
\usepackage{longtable}
\usepackage{tabularx}
\newcolumntype{Y}{>{\raggedright\arraybackslash}X}
\usepackage{array}
\usepackage{enumitem}
\usepackage{csquotes}
\usepackage[printonlyused]{acronym}
\usepackage{tocbasic} % better control of ToC lists
\usepackage{fancyhdr}
\usepackage{lastpage}
\usepackage{titling}
\usepackage{amssymb}
\usepackage{amsmath}
\usepackage{siunitx}
\usepackage{float}
\usepackage{pgf-pie}
% TikZ for pipeline diagrams used in chapters
\usepackage{tikz}
\usetikzlibrary{shapes.geometric, arrows, positioning}
\setlength{\parindent}{0pt} 
\setlength{\parskip}{6pt}

% ---------- Page Setup ----------
\geometry{left=30mm,right=20mm,top=20mm,bottom=30mm}
\onehalfspacing
\setkomafont{sectioning}{\normalfont\bfseries}
\setkomafont{disposition}{\normalfont\bfseries}
\renewcommand{\arraystretch}{1.2}

% ---------- Header / Footer ----------
\pagestyle{fancy}
\fancyhf{}
\fancyfoot[C]{\thepage\ / \pageref{LastPage}}
\renewcommand{\headrulewidth}{0pt}

% --- Seitenränder ---
\usepackage[a4paper, left=3cm, right=2cm, top=2cm, bottom=3cm]{geometry}

% --- Kopf- und Fußzeilen ---
\usepackage{fancyhdr}
\usepackage{graphicx} % Für das Logo
\usepackage{lastpage} % Für "Seite X von Y" falls benötigt

% --- Stil für alle Seiten ---
\pagestyle{fancy}

% Kopfzeile leeren
\fancyhead{}
\fancyfoot{}

% --- Kopfzeile: links = Kapitelname, rechts = Logo ---
\fancyhead[L]{\nouppercase{\leftmark}} % aktueller Kapitelname
\fancyhead[R]{\includegraphics[height=1cm]{Bilder/TH_Nuernberg_Logo.png}} % passe den Pfad an!

% --- Fußzeile: Seitenzahlen zentriert ---
\fancyfoot[C]{\thepage}

% --- Linien unter/über Kopf- und Fußzeile ---
\renewcommand{\headrulewidth}{0.4pt}
\renewcommand{\footrulewidth}{0pt}

% --- Optional: Stil für Kapitelanfangsseiten (ohne Logo) ---
\fancypagestyle{plain}{
  \fancyhead{}
  \fancyfoot[C]{\thepage}
  \renewcommand{\headrulewidth}{0pt}
}

\title{"BA Nodem Build"}
\author{ }
\date{\today}

%-----------Literatur--------------
%-----------Literatur--------------
\usepackage[backend=biber,style=authoryear,sorting=nyt,maxbibnames=99]{biblatex}
\addbibresource{literatur.bib} % nom exact du .bib

%--Document can have many numberedd subsection---
\setcounter{secnumdepth}{3}
\setcounter{tocdepth}{3} % (optionnel : pour qu'elles apparaissent aussi dans la table des matières)

% --- TikZ für Flussdiagramme ---
\usepackage[utf8]{inputenc}  % encodage
\usepackage[T1]{fontenc}     % caractères accentués
\usepackage[ngerman]{babel}  % ou [french] selon la langue
\usepackage{tikz}
\usetikzlibrary{shapes.geometric, arrows, positioning}

% --- Stile für Flowchart-Knoten und Pfeile ---
\tikzset{
  startstop/.style = {rectangle, rounded corners, minimum width=3.2cm,
    minimum height=1.0cm, text centered, draw=black, fill=red!30},
  process/.style   = {rectangle, minimum width=3.6cm, minimum height=1.0cm,
    text centered, draw=black, fill=orange!30},
  arrow/.style     = {thick,->,>=stealth}
}

% ---------- Document ----------
\begin{document}
\pagenumbering{Roman}
\tableofcontents
\cleardoublepage

\listoffigures
\addcontentsline{toc}{chapter}{Abbildungsverzeichnis}
\cleardoublepage

\listoftables
\addcontentsline{toc}{chapter}{Tabellenverzeichnis}
\cleardoublepage

\chapter*{Abkürzungsverzeichnis}
\addcontentsline{toc}{chapter}{Abkürzungsverzeichnis}
\begin{acronym}[LiDAR]
    \acro{LiDAR}{Light Detection and Ranging}
    \acro{ROS}{Robot Operating System}
    \acro{ToF}{Time of Flight}
    % \acro{PCL}{Point Cloud Library}
    % weitere Abkürzungen hier ergänzen
\end{acronym}
\cleardoublepage

\pagenumbering{arabic}

% Include the chapter content
\chapter{Einleitung}

\section{Hinführung}
Autonome und hochautomatisierte Fahrzeuge sowie mobile Robotersysteme sind auf eine präzise, robuste und nachvollziehbare Umfelderkennung angewiesen, um Sicherheit zu gewährleisten und betriebliche Leistungsfähigkeit zu erreichen. Unter den verfügbaren Sensortechnologien liefert LiDAR dichte, dreidimensionale Punktwolken mit hoher Winkelauflösung und weitgehend konstanter Leistungsfähigkeit über verschiedene Beleuchtungsbedingungen hinweg. Im Kontext des automatisierten Fahrens hat sich LiDAR als zentraler Baustein der Wahrnehmung etabliert und bildet die Grundlage zahlreicher Detektions-, Segmentierungs- und Trackingverfahren \parencite{Arnold2019Survey}.

Aktuelle 360\textdegree{}-LiDAR-Sensoren wie der Ouster~OS1 vereinen hohe Winkelauflösung mit großer Reichweite und erzeugen Punktwolkenströme im Millionenbereich pro Sekunde. Dies eröffnet differenzierte Möglichkeiten zur Modellierung der Umgebung, stellt jedoch zugleich hohe Anforderungen an die effiziente Vorverarbeitung, die robuste Bodensegmentierung, die objektspezifische Gruppierung sowie die zuverlässige Verfolgung in (nahezu) Echtzeit. Für eine systematische, reproduzierbare Integration bietet ROS~2 eine modulare Middleware mit klaren Kommunikationssemantiken und praxiserprobten Werkzeugketten \parencite{Macenski2022ROS2,ROS2Docs}. Die in dieser Arbeit betrachtete Sensorplattform basiert auf einem Ouster~OS1 und entspricht einem typischen Automotive‑Setup \parencite{OusterOS1}.

\section{Themenspezifizierung}
Ziel dieser Arbeit ist die Entwicklung und Evaluierung einer modularen, echtzeitnahen Pipeline zur LiDAR‑basierten Umfelderkennung auf Basis hochauflösender 360\textdegree{}‑Daten. Im Fokus stehen (i) die qualitätssteigernde Vorverarbeitung (u.\,a. Ausreißerreduktion, Downsampling), (ii) eine robuste Bodensegmentierung für unterschiedliche Szenarien, (iii) die extraktionssichere Bildung von Clustern und zugehörigen Bounding‑Boxen sowie (iv) die objektspezifische Verfolgung. Die Implementierung erfolgt in ROS~2 mit dem Ziel, klare Schnittstellen, reproduzierbare Experimente und eine messbare Laufzeitgüte zu erzielen. Die Beurteilung erfolgt anhand etablierter Metriken (z.\,B. IoU für Boden/Nicht‑Boden, Precision/Recall für Detektion, Latenz/FPS für Laufzeit) und orientiert sich an in der Literatur und in Benchmarks üblichen Bewertungskonzepten \parencite{Arnold2019Survey,KITTI2012}.

\section{Abgrenzung}
Nicht Bestandteil der Arbeit sind Sensorfusion mit Kamera/Radar, semantische Segmentierung mittels großskaliger neuronaler Netze inklusive umfangreicher Trainingsläufe, sowie kartengestützte Lokalisierung und globale Trajektorienplanung. Ebenfalls außerhalb des Fokus liegen Hardware‑Design und optoelektronische Auslegung des Sensors, eine sicherheitstechnische Zertifizierung sowie herstellerübergreifende Performancevergleiche jenseits der eingesetzten Ouster‑OS1‑Konfiguration. Kalibrierung wird auf die für die Verarbeitung erforderliche Extrinsik/Zeitsynchronisation begrenzt; eine detaillierte Analyse der Langzeitdrift ist nicht Gegenstand.

\section{Forschungsfragen}
Zur Zielerreichung werden u.\,a. folgende Fragen untersucht:
\begin{enumerate}[label=\textbf{F\arabic*})]
  \item Welche Vorverarbeitungsstrategien verbessern die Datenqualität (z.\,B. Rausch‑/Ausreißerreduktion, Downsampling) bei vertretbarer Laufzeit und stabiler End‑zu‑Ende‑Latenz?
  \item Welche Verfahren zur Bodensegmentierung (z.\,B. RANSAC‑Ebene, progressive morphologische Filter, Cloth‑Simulation) erreichen die beste Balance aus Genauigkeit (z.\,B. IoU) und Robustheit über Szenarien hinweg \parencite{FischlerBolles1981,Zhang2003PMF,Zhang2016CSF}? 
  \item Wie wirken sich Vorverarbeitung und Bodensegmentierung auf die nachgelagerte Objektbildung (Clustering/Bounding‑Boxes) und Detektion aus (z.\,B. Precision/Recall)?
  \item Wie lässt sich die Pipeline in ROS~2 so integrieren, dass deterministische Verarbeitung und geforderte Durchsatzraten (FPS) erreicht werden, ohne die Erkennungsleistung signifikant zu kompromittieren?
\end{enumerate}

\section{Beitrag der Arbeit}
Die Arbeit leistet folgende Beiträge:
\begin{itemize}
  \item Entwurf und Implementierung einer modularen ROS~2‑Pipeline für Vorverarbeitung, Bodensegmentierung, Cluster‑ und Box‑Bildung sowie Tracking auf Basis eines Ouster~OS1‑Sensors.
  \item Systematische Gegenüberstellung ausgewählter Boden‑ und Clusterverfahren unter konsistenten Metriken (Genauigkeit/Robustheit vs. Laufzeit), inklusive Analyse relevanter Parameterwirkungen.
  \item Bereitstellung reproduzierbarer Experimente mit klar dokumentierten Konfigurationen, Datenschnitten und Auswertungsskripten.
\end{itemize}

\section{Methodische Vorgehensweise}
Die Bearbeitung folgt einem experimentell‑analytischen Vorgehen: (i) Literaturrecherche und Anforderungsanalyse, (ii) Datenerhebung/‑auswahl mit Ouster~OS1, (iii) Implementierung der Pipelinebausteine, (iv) Integration in ROS~2 mit definierten QoS‑Profilen und Ausführungsmodellen, (v) experimentelle Evaluierung anhand festgelegter Metriken und Lastfälle, (vi) Diskussion der Trade‑offs zwischen Genauigkeit, Robustheit und Laufzeit.

\section{Aufbau der Arbeit}
Kapitel~\ref{chap:stand-der-technik} stellt wissenschaftlich‑technische Grundlagen, Messprinzipien und sensorische Eigenschaften dar und verortet die gewählten Verfahren im Stand der Technik. Kapitel~\ref{chap:systemarchitektur} beschreibt die Systemarchitektur, Datenflüsse und Implementierungsentscheidungen in ROS~2. Darauf aufbauend folgen die Kapitel zur Vorverarbeitung, zur Cluster‑/Bounding‑Box‑Bildung und zur Objektverfolgung mit jeweils fokussierten Evaluationsabschnitten. Ein abschließendes Kapitel fasst die Ergebnisse zusammen und gibt einen Ausblick auf weiterführende Arbeiten.


\chapter{Stand der Technik und Wissenschaft}
\label{chap:stand-der-technik}

LiDAR hat sich in modernen Fahrerassistenzsystemen (ADAS) und beim automatisierten Fahren als zentrale Sensortechnologie etabliert. Im Vergleich zu Kameras (empfindlich gegenüber Beleuchtung/Blendung) und Radar (robust, jedoch begrenzte Winkelauflösung) liefert LiDAR hochgenaue Distanzinformationen mit hoher Winkelauflösung als dreidimensionale Punktwolken. Historisch stammt LiDAR aus Geodäsie, Fernerkundung und Atmosphärenforschung und fand mit Fortschritten in Laser- und Optoelektronik seit den 2000er‑Jahren zunehmend Eingang in die Automobiltechnik. Ziel dieses Kapitels ist die Darstellung der wissenschaftlich‑technischen Grundlagen, relevanter Messprinzipien und Anwendungen sowie der in dieser Arbeit verwendeten Sensorik.

\section{Sensoraufbau}
\label{sec:sensoraufbau}
Ein LiDAR-Sensor umfasst Sendeoptik und Strahlablenkung, Empfangsoptik und Lichtsensor sowie Auslese- und Steuerungselektronik.

\subsection{Sendeoptik und Strahlablenkung}
Die Sendeoptik formt das Intensitätsprofil und beeinflusst damit Reichweite, Divergenz und Winkelauflösung. Refraktive Linsen oder diffraktive optische Elemente (DOE) erzeugen die gewünschte Strahlform. Rasternde Systeme nutzen Spiegel/Prismen (rotierend/oszillierend) zur Ablenkung und sukzessiven Abtastung; Flash‑Systeme beleuchten die Szene pro Aufnahme vollständig (z.\,B. mit Diffusoren/DOE). Die Strahlablenkung ist das zentrale Element zur Erzeugung der Winkelinformation über den erfassten Raumwinkel.

\subsection{Empfangsoptik und Lichtsensor}
Die Empfangsoptik sammelt das rückgestreute Licht; Apertur und Transmission bestimmen das Signal gemäß der LiDAR‑Gleichung. Antireflexbeschichtungen und auf die Quelle abgestimmte Durchlässigkeit reduzieren Rauschen durch Umgebungslicht. Als Detektoren kommen typischerweise Avalanche‑Photodioden (APD) oder SPAD‑Arrays zum Einsatz; sie bieten hohe Empfindlichkeit und kurze Anstiegszeiten für schwache, kurze Pulse.

\subsection{Auslese- und Steuerungselektronik}
Verstärker, ADC/TDC und Multiplexer digitalisieren die Signale und bestimmen die Laufzeit je Messpunkt. Moderne Sensoren berechnen Distanzwerte auf Sensorebene und übernehmen Vorverarbeitung (z.\,B. Mehrfach‑Echo, Intensität). Die Steuerung selektiert aktive Zeitfenster (Reduktion von Umgebungslicht), synchronisiert Ablenkung und Detektion und ordnet Messpunkte den Raumwinkeln zu.

\section{Wellenlängenbereiche und Augensicherheit}
\label{sec:wellenlaenge}
Automotive LiDAR arbeitet überwiegend im nahen Infrarot (NIR): \(\sim\)\SIrange{870}{950}{\nano\meter} (Si‑Detektoren) sowie \SI{1064}{\nano\meter} oder \SI{1550}{\nano\meter} (InGaAs‑Detektoren). 905\,nm‑Systeme sind kosteneffizient, profitieren von verfügbarer Halbleitertechnologie und ausreichender Detektorempfindlichkeit; 1550\,nm erlaubt innerhalb der IEC~60825‑1‑Grenzen höhere Sendeleistungen (geringere Netzhautgefährdung), erfordert jedoch teurere Detektoren und ggf. aufwändigere Optik. Atmosphärische Fenster, solare Spektralanteile und Streuung an Partikeln (Mie‑Streuung) beeinflussen Reichweite und Robustheit bei Regen/Nebel \parencite{Amann2001LaserRanging}.

\section{Messprinzipien}
\label{sec:messprinzip}
\subsection{Direct Time‑of‑Flight (dToF)}
Beim dToF wird ein kurzer Laserpuls ausgesendet, am Objekt reflektiert und mit Lichtgeschwindigkeit \(c\) detektiert. Aus der gemessenen Zeitdifferenz \(\Delta t\) zwischen Emission und Detektion ergibt sich die Entfernung \(d\):
\begin{equation}
  d = \frac{c \cdot \Delta t}{2}
\end{equation}
Der Faktor \(\tfrac{1}{2}\) berücksichtigt Hin‑ und Rückweg \parencite{Amann2001LaserRanging}.

\subsection{Indirect ToF (iToF)}
iToF bestimmt Distanzen über die Phasenverschiebung einer modulierten Dauerlichtquelle. Vorteile sind kompakte, teils kostengünstige Sensorik und gute Kurzstreckenpräzision; Grenzen liegen in Ambiguität (Mehrdeutigkeit) und erhöhter Multipath‑Empfindlichkeit \parencite{Amann2001LaserRanging}.

\subsection{FMCW‑LiDAR}
Frequenzmoduliertes Dauerstrich‑LiDAR (FMCW) misst Entfernung und Relativgeschwindigkeit (Doppler) über Frequenzchirps. Vorteile sind Interferenzresistenz und gleichzeitige Geschwindigkeitsmessung; demgegenüber stehen höhere Auswerte‑ und Hardwarekomplexität \parencite{Behroozpour2017LidarArch}.

\subsection{Fehlerquellen}
Wesentliche Fehlerquellen sind Zeitjitter (TDC/ADC), Multipath, Einfallswinkelabhängigkeit, Strahldivergenz, Oberflächenreflektivität, Temperatur‑/Drifteinflüsse sowie mechanischer Jitter (Drehzahl). Sie beeinflussen Reichweite, Präzision und die Zuverlässigkeit der Segmentierung/Klassifikation.

\section{Typische Anwendungsszenarien im Fahrzeugumfeld}
\label{sec:anwendung}
LiDAR unterstützt zentrale ADAS‑Funktionen (AEB, ACC, LKA), Freiraum‑/Belegtheitsdetektion, Hinderniserkennung sowie hochautomatisierte Fahrfunktionen (3D‑Umfeldmodellierung, SLAM, Lokalisierung, Baustellen‑/Engstellenmanagement). In der Sensorsuite ergänzt LiDAR Kamera und Radar durch präzise 3D‑Geometrie. Limitationen entstehen bei Nebel/Regen, geringer Oberflächenreflektivität und Interferenzen; robuste Fusion und fehlertolerante Algorithmen sind daher notwendig. Leistungsanforderungen umfassen definierte Bildraten, End‑zu‑Ende‑Latenzen und Robustheitsmetriken \parencite{Arnold2019Survey}.

\section{Algorithmischer Stand der Technik}
Für LiDAR‑Umfelderkennung werden typischerweise folgende Bausteine kombiniert:
\begin{itemize}
  \item \textbf{Vorverarbeitung:} Downsampling (z.\,B. \emph{VoxelGrid}), Ausreißerfilter (\emph{Statistical/Radius Outlier Removal}), ggf. Bewegungsentzerrung, Intensitätsnormalisierung, Polarkoordinaten/BEV‑Darstellung, Mehrfach‑Echo‑Nutzung.
  \item \textbf{Bodensegmentierung:} RANSAC‑Ebene \parencite{FischlerBolles1981}, Progressive Morphological Filter (PMF) \parencite{Zhang2003PMF}, gitterbasierte Höhenkarten, Cloth Simulation Filtering (CSF) \parencite{Zhang2016CSF}, probabilistische Modelle (z.\,B. Gaussian Processes); robust gegenüber Neigungen/Bordsteinen.
  \item \textbf{Clustering:} Euklidische Clusterung, DBSCAN/HDBSCAN \parencite{Ester1996DBSCAN}, Region Growing; Parameter (Epsilon/Radius, Mindestpunkte) steuern Auflösung und Rauschtoleranz.
  \item \textbf{Detektion/Klassifikation:} Geometrische Merkmale (PCA, OBB, Formfeatures) oder lernbasierte 3D‑Detektoren (z.\,B. Pillars/SECOND/CenterPoint), BEV‑Pipelines; Kompromiss zwischen Datenbedarf, Robustheit und Laufzeit.
  \item \textbf{Tracking:} KF/EKF/UKF, JPDAF/MHT sowie praxisnahe Multi‑Target‑Pipelines (Datenassoziation per NN/IoU, Geburt/Tod‑Logik, Bewegungsmodelle).
\end{itemize}

\section{Ouster OS1 am Opel Astra der TH Nürnberg}
\subsection{Spezifische Eigenschaften des Ouster OS1}
Der OS1 ist ein 360\textdegree{}‑LiDAR mittlerer Reichweite. Er arbeitet bei \(\approx\)\SI{865}{\nano\meter} mit \(360\textdegree{}\) horizontalem und \(\approx\)\SI{42.4}{\degree} vertikalem Sichtfeld. Je nach Konfiguration sind 32/64/128 Vertikalkanäle verfügbar; horizontal \(512/1024/2048\) Messpunkte pro Umdrehung. Je nach Modus entstehen \(\sim\)\(1.3\)–\(5.2\)\,M Punkte/s. Typische Reichweiten liegen bis \(\sim\)\SI{170}{\meter} (hohe Reflektivität) bzw. \(\sim\)\SI{90}{\meter} (dunkle Oberflächen, starke Sonne). Distanzgenauigkeit \(\approx\)\(\pm\)\SI{2.5}{\centi\meter}, typische Präzision \(\pm\)\SIrange{0.5}{3}{\centi\meter}. Bis zu zwei Echos pro Puls werden registriert. Datenfelder umfassen Distanz, Intensität/Reflexivität, Kanalnummer, Azimut und Zeitstempel; zusätzlich liefert eine IMU Bewegungsdaten mit \SI{100}{\hertz}. Die Ausgabe erfolgt über Gigabit‑Ethernet (UDP), das Gehäuse ist nach IP68/IP69K geschützt; Betriebstemperatur von \(-\)\SI{40}{\celsius} bis \(+\)\SI{60}{\celsius} \parencite{OusterOS1}.

\subsection{Sensorpositionierung am Opel Astra}
Für die Fahrzeugintegration sind Montageort (Dachmitte), Aufbauhöhe und Neigung so zu wählen, dass das vertikale Sichtfeld den Nahbereich abdeckt und Fahrzeugteile nicht occludieren. Die Extrinsik wird relativ zu \texttt{base\_link} bestimmt (z.\,B. Zieltafel/ICP) und über \textit{tf2} bereitgestellt; relevante Frames sind \texttt{map}, \texttt{odom}, \texttt{base\_link}, \texttt{os\_sensor}/\texttt{os\_lidar}. Die Zeitbasis wird vorzugsweise per PTP synchronisiert. Messunsicherheiten (Kalibrierung, Synchronisation) wirken direkt auf Segmentierung, Detektion und Tracking und sind in der Bewertung zu berücksichtigen.

\section{Bezug zu den Forschungsfragen}
Die Auswahl und Ausgestaltung der Algorithmen richtet sich an den Anforderungen der Kapitel zur Vorverarbeitung, Cluster‑/Bounding‑Box‑Bildung und Objektverfolgung aus. Bewertet werden Genauigkeit, Robustheit und Echtzeitfähigkeit (End‑zu‑Ende‑Latenz, Ressourcenbedarf) in repräsentativen Szenarien mit OS1‑Daten. Die hier beschriebenen Eigenschaften des Sensors und der Messprinzipien begründen die methodischen Entscheidungen in den folgenden Kapiteln.

\chapter{Systemarchitektur}
\label{chap:systemarchitektur}

Dieses Kapitel beschreibt die technische Systemarchitektur der Umfelderkennung auf Basis von ROS~2 und eines Ouster~OS1. Ziel ist eine nachvollziehbare, konsistente und echtzeitfähige Verarbeitungskette. Die Darstellung umfasst (i) Grundlagen von ROS~2, (ii) Datenrepräsentation und Bibliotheken, (iii) Arbeitsumgebung, (iv) Gesamtsystem mit Datenaufnahme und Pipeline, (v) Deployment sowie (vi) Bewertungsmetriken.

In diesem Zusammenhang werden die Software- und Hardwarekomponenten beschrieben, die für die Erfassung, Verarbeitung und Auswertung der Sensordaten verwendet wurden. Zunächst werden die Grundlagen des Robot Operating System~2~(ROS~2) erläutert, das als Middleware zur Kommunikation zwischen den einzelnen Modulen dient. Anschließend wird die unter Linux~Ubuntu~22.04 eingerichtete Arbeitsumgebung vorgestellt, in der die Implementierung und Tests durchgeführt wurden. Abschließend erfolgt ein Überblick über das Gesamtsystem mit den definierten Anforderungen, der Datenerfassung und der praktischen Umsetzung der entwickelten Messkette.

\section{Grundlagen von ROS~2}
\label{sec:ros2_basics}
Das \emph{Robot Operating System~2 (ROS~2)} ist ein quelloffenes Framework für modulare, verteilte Robotiksysteme. Es stellt Middleware-basierte Kommunikation (DDS), wiederverwendbare Komponenten und Entwicklungswerkzeuge bereit. Dank Skalierbarkeit und Plattformunabhängigkeit wird ROS~2 in Forschung und Industrie eingesetzt (z.\,B. autonome Fahrzeuge, mobile Robotik, Inspektion).

\subsection{Nodes}
\label{sec:ros2_nodes}
Nodes sind die grundlegenden Prozesse eines ROS~2-Systems (z.\,B. Sensordatenerfassung, Verarbeitung, Aktorik). Die Modularisierung erlaubt klar definierte Schnittstellen und Wiederverwendung (vgl. \cite{ros2_docs}).

\subsection{Topics}
\label{sec:ros2_topics}
Themenkanäle (Topics) transportieren Nachrichten nach dem Publish/Subscribe-Prinzip. Publisher und Subscriber sind entkoppelt, was flexible, verteilte Architekturen ermöglicht (vgl. \cite{ros2_docs}).

\subsection{Nachrichten}
\label{sec:ros2_messages}
Nachrichten sind wohldefinierte Datenstrukturen (z.\,B. Ganzzahlen, Fließkomma, Arrays, benutzerdefinierte Typen) und erlauben einen sprach- und implementationsübergreifenden Austausch (vgl. \cite{ros2_docs}).

\subsection{Bags}
\label{sec:ros2_bags}
\emph{ROS~2 Bags} zeichnen Topic-Nachrichten auf und spielen sie wieder ab. Sie unterstützen Analyse, Debugging und reproduzierbare Experimente (vgl. \cite{ros2_docs}).

\subsection{RViz2}
\label{sec:ros2_rviz2}
\emph{RViz2} visualisiert Punktwolken, Trajektorien und Systemzustände in 3D und unterstützt Entwicklung und Fehlersuche (vgl. \cite{ros2_docs}).

\subsection{DDS, Discovery und Transports}
ROS~2 nutzt DDS als Middleware. In \emph{Humble} sind \emph{CycloneDDS} und \emph{FastDDS} verbreitet (RMW: \texttt{rmw\_cyclonedds}, \texttt{rmw\_fastrtps}).
\begin{itemize}
  \item \textbf{Discovery}: Teilnehmer-/Endpunkt-Discovery (SPDP/SEDP) über Multicast/Unicast; Domänen-ID segmentiert Netze.
  \item \textbf{Transports}: UDP als Standard; Shared-Memory (SHM) und \emph{intra-process} reduzieren Kopierkosten bei Co-Lokation.
  \item \textbf{Konfiguration}: DDS-XML/Umgebungsvariablen zur Steuerung von Interfaces, Whitelists, Timern und Puffergrößen.
\end{itemize}
Empfehlung: SHM/\emph{intra-process} aktivieren, Netzwerkschnittstelle gezielt wählen (Sensor-LAN), Domänen trennen.

\subsection{Executors, Callback-Gruppen und Intra-Process}
Die Abarbeitung erfolgt über Executor-Konzepte:
\begin{itemize}
  \item \textbf{SingleThreadedExecutor}: deterministische Reihenfolge, für lineare Pipelines geeignet.
  \item \textbf{MultiThreadedExecutor}: parallele Verarbeitung über \emph{CallbackGroups} (reentrant/exclusive) für rechenintensive Schritte.
  \item \textbf{Intra-Process-Comms}: \texttt{NodeOptions\{use\_intra\_process\_comms=true\}} reduziert Kopien großer Nachrichten (Punktwolken).
\end{itemize}
CPU-Affinitäten/Prioritäten und abgestimmte Puffer (QoS-Depth) minimieren Latenzjitter.

\subsection{Quality of Service (QoS)}
\label{sec:ros2_qos}
QoS-Profile in ROS~2 definieren Kommunikationssemantik (Zuverlässigkeit, Pufferung, Haltbarkeit, zeitliche Garantien) und beeinflussen Latenz, Verlustverhalten und Speicherbedarf. Für hochratige Sensorströme sind konsistente Profile entlang der Pipeline wesentlich.

\paragraph{Policies und Parameter}
\begin{itemize}
  \item \textbf{Reliability} (Reliable/BestEffort): \emph{Reliable} minimiert Paketverluste (unter Last potentiell höhere Latenz); \emph{BestEffort} priorisiert geringe Latenz und toleriert Verluste.
  \item \textbf{History/Depth}: \emph{KeepLast} mit begrenzter Tiefe limitiert Puffer und Latenz; \emph{KeepAll} ist für Sensorstreams ungeeignet.
  \item \textbf{Durability}: \emph{Volatile} für Sensordaten; \emph{TransientLocal} für statische Konfigurationen (z.\,B. Kalibrierung).
  \item \textbf{Deadline/Liveliness}: Überwachung von Aktualisierungsraten bzw. Lebenszeichen.
  \item \textbf{Lifespan}: Verfallszeit für Nachrichten (selten bei LiDAR erforderlich).
\end{itemize}

\paragraph{Empfohlene Profile}
\begin{itemize}
  \item Verarbeitungspipeline: \emph{Reliable} + \emph{KeepLast}(5--10), \emph{Volatile} (verlustarm bei kontrollierter Pufferung).
  \item Visualisierung (RViz2): \emph{BestEffort} + \emph{KeepLast}(1--5), \emph{Volatile} (niedrige Latenz, verlusttolerant).
\end{itemize}

\paragraph{Zuordnung zu Topics}
\begin{itemize}
  \item \texttt{/ouster/points}: Reliable, KeepLast(10), Volatile.
  \item \texttt{/points\_cropped}, \texttt{/points\_voxel}: Reliable, KeepLast(10), Volatile.
  \item \texttt{/obstacle\_points}: Reliable, KeepLast(10), Volatile.
  \item \texttt{/detections\_raw}, \texttt{/tracks\_raw}: Reliable, KeepLast(5), Volatile.
\end{itemize}

\paragraph{Übersicht und Beispiele}
\begin{table}[h!]
  \centering
  \begin{tabular}{l l l l}
    \hline
    \textbf{Topic} & \textbf{Reliability} & \textbf{History/Depth} & \textbf{Durability} \\
    \hline
    /ouster/points & Reliable & KeepLast/10 & Volatile \\
    /points\_cropped & Reliable & KeepLast/10 & Volatile \\
    /points\_voxel & Reliable & KeepLast/10 & Volatile \\
    /obstacle\_points & Reliable & KeepLast/10 & Volatile \\
    /detections\_raw & Reliable & KeepLast/5 & Volatile \\
    /tracks\_raw & Reliable & KeepLast/5 & Volatile \\
    \hline
  \end{tabular}
  \caption{Empfohlene QoS-Profile für zentrale Topics.}
  \label{tab:qos-topics}
\end{table}

QoS-Override bei Wiedergabe (verkürzte YAML-Struktur):
\begin{verbatim}
topics:
  - topic: /ouster/points
    qos:
      reliability: reliable
      durability: volatile
      history: keep_last
      depth: 10
  - topic: /tracks_raw
    qos:
      reliability: reliable
      durability: volatile
      history: keep_last
      depth: 5
\end{verbatim}

\paragraph{QoS-Aushandlung und Fehlermodi}
Publisher/Subscriber handeln effektive QoS aus. Sonderfälle: (i) \emph{Reliable}~Publisher und \emph{BestEffort}~Subscriber: Verbindung möglich, Abonnent erhält \emph{BestEffort}; Publisher-Puffer können wachsen. (ii) Mismatch bei \emph{Durability}: Keine Nachlieferung historischer Daten für \emph{Volatile}. (iii) Zu geringe \emph{Depth}: Drops oder ansteigende E2E-Latenz (Backpressure). Empfehlung: QoS pro Topic explizit festlegen und monitoren.

Beispiel C++-QoS-Konfiguration:
\begin{verbatim}
// Verarbeitung: verlustarm, begrenzte Puffer
rclcpp::QoS qos_proc(rclcpp::KeepLast(10));
qos_proc.reliability(RMW_QOS_POLICY_RELIABILITY_RELIABLE);
qos_proc.durability(RMW_QOS_POLICY_DURABILITY_VOLATILE);
auto pub = node->create_publisher<sensor_msgs::msg::PointCloud2>(
  "/points_voxel", qos_proc);

// Visualisierung: BestEffort
rclcpp::QoS qos_viz(rclcpp::KeepLast(3));
qos_viz.best_effort();
auto sub = node->create_subscription<sensor_msgs::msg::PointCloud2>(
  "/obstacle_points", qos_viz,
  [](sensor_msgs::msg::PointCloud2::SharedPtr){ /* ... */ });
\end{verbatim}

\section{Datenrepräsentation und Bibliotheken}
\subsection{\texttt{sensor\_msgs/PointCloud2}: Felder und Konventionen}
Ouster~OS1-Daten werden als \texttt{sensor\_msgs/PointCloud2} publiziert. Wichtige Felder und Semantik:
\begin{itemize}
  \item \texttt{x, y, z}: kartesische Koordinaten [m] im \texttt{frame\_id} (Sensor- oder Fahrzeugrahmen; konsistent mit \textit{tf2}).
  \item \texttt{intensity}: reflektierte Signalstärke (treiberspezifische Skalierung); vergleichbar innerhalb eines Scans.
  \item \texttt{ring}: vertikaler Kanalindex; erlaubt kanalspezifische Filter/Strukturierung (organisierte Wolken nach Ringen).
  \item \texttt{time} (falls vorhanden): Punktzeitversatz relativ zu \texttt{header.stamp} (intra-scan motion compensation).
  \item Hinweis: konkrete Feldtypen (z.\,B. \texttt{float32}, \texttt{uint16}) sind treiber-/konfigurationsabhängig.
\end{itemize}
Kodierung: in der Regel little-endian (\texttt{is\_bigendian = false}). Speicherlayout: linearer Puffer \texttt{data} mit \texttt{point\_step}/\texttt{row\_step}; \texttt{width}/\texttt{height} definieren die 2D-Anordnung (rotierender LiDAR: \texttt{height} = Ringanzahl, \texttt{width} = Punkte/Umdrehung). \texttt{is\_dense} signalisiert \enquote{NaN}-Vorkommen. Organisierte Wolken (Ring~\(\times\) Azimut) erleichtern nachbarschaftsbasierte Filter/Segmentierung.

\subsection*{Feldmetadaten und Offsets}
Die Einträge in \texttt{fields[]} beschreiben Name, Offset, Typ und Zählung. Beispiel (verkürzt, treiberspezifisch):
\begin{verbatim}
fields:
  - {name: x,         offset: 0,  datatype: FLOAT32}
  - {name: y,         offset: 4,  datatype: FLOAT32}
  - {name: z,         offset: 8,  datatype: FLOAT32}
  - {name: intensity, offset: 12, datatype: UINT16}
  - {name: ring,      offset: 14, datatype: UINT16}
point_step: 16
\end{verbatim}

\subsection*{Zugriff über Iteratoren}
Robuster Feldzugriff mit Iterator-API vermeidet Zeigerarithmetik:
\begin{verbatim}
#include <sensor_msgs/point_cloud2_iterator.hpp>
for (sensor_msgs::PointCloud2ConstIterator<float> it_x(msg, "x"), it_y(msg, "y"), it_z(msg, "z");
     it_x != it_x.end(); ++it_x, ++it_y, ++it_z) {
  const float x = *it_x, y = *it_y, z = *it_z;
}
\end{verbatim}

\subsection{Point Cloud Library (PCL)}
\label{chap:pcl}
Die \emph{Point Cloud Library (PCL)} bildet die Grundlage der Umfelderkennung. Sie stellt effiziente Verfahren zur Filterung, Segmentierung, Clusterbildung und geometrischen Modellierung bereit (vgl. \cite{pcl_docs_2025}). In ROS~2 erfolgt die Einbindung über \texttt{pcl\_conversions} und ggf. \texttt{pcl\_ros}; \texttt{sensor\_msgs/PointCloud2} wird bidirektional in \texttt{pcl::PointCloud<T>} konvertiert.

\subsection*{Genutzte Module und Algorithmen}
\begin{itemize}
  \item \textbf{Filter}: \texttt{VoxelGrid}, \texttt{PassThrough}/\texttt{CropBox}, \texttt{StatisticalOutlierRemoval}.
  \item \textbf{Segmentierung}: Planarsegmentierung (RANSAC) für Boden.
  \item \textbf{Clustering}: \texttt{EuclideanClusterExtraction}.
  \item \textbf{Konvertierung}: \texttt{pcl::fromROSMsg}/\texttt{pcl::toROSMsg}, \texttt{pcl\_conversions}.
\end{itemize}
\textbf{Performance}: Vorallokation, Kopiervermeidung, geeignete Punkttypen (z.\,B. \texttt{pcl::PointXYZI}).

\section{Arbeitsumgebung: Ubuntu 22.04}
Für Entwicklung und Ausführung wird \textbf{Ubuntu~22.04~LTS (Jammy Jellyfish)} verwendet. Als Referenzplattform für ROS~2 ermöglicht Ubuntu eine nahtlose Integration von Bibliotheken, Treibern und Werkzeugen (vgl. \cite{ubuntu_docs_2025}). Die Distribution bietet stabile C++/Python-Toolchains, hohe Sicherheit und breite Unterstützung in wissenschaftlicher wie industrieller Softwareentwicklung. Die enge Verzahnung mit der in dieser Arbeit eingesetzten ROS-Distribution \emph{Humble~Hawksbill} vereinfacht die Einrichtung der Abhängigkeiten. Die aktive Entwicklergemeinschaft sorgt durch regelmäßige Updates, umfassende Dokumentation und eine große Auswahl an Open-Source-Paketen für einen reibungslosen Entwicklungsprozess. Dank der modularen Struktur von Linux lässt sich die Arbeitsumgebung flexibel an die spezifischen Anforderungen der Sensorintegration und der ROS-Module anpassen.

\subsection*{Terminator-Terminalemulator}
Der Terminalemulator \emph{Terminator} ermöglicht die parallele Ausführung und Überwachung mehrerer ROS~2-Knoten in aufgeteilten Terminalflächen. Dadurch können unterschiedliche Prozesse simultan beobachtet und kontrolliert werden (z.\,B. Sensorknoten, Filtermodule, Visualisierung). Wiederkehrende Layouts lassen sich speichern, wodurch eine standardisierte Entwicklungsumgebung entsteht, die den Überblick über komplexe Systemprozesse erleichtert und den Arbeitsfluss optimiert.

\subsection*{Bash-Skripte}
Bash-Skripte automatisieren wiederkehrende Aufgaben (Workspace laden, Launch starten, Aufzeichnung) und sind ein zentrales Werkzeug in Linux-basierten Entwicklungsumgebungen. In dieser Arbeit fungiert ein Skript als Bindeglied zwischen der GUI und dem Python-\texttt{launch}-File der Pipeline: Beim Start werden die ROS~2-Umgebung geladen, die relevanten Workspaces gesourced und das zentrale Launch-File ausgeführt, das alle erforderlichen Knoten in definierter Reihenfolge aktiviert.

\subsection*{ROS~2 CLI und Monitoring}
Zur Inspektion der Laufzeitumgebung werden \texttt{ros2}-Werkzeuge genutzt: \texttt{topic list/info/hz/bw/delay}, \texttt{node list/info}, \texttt{interface show}, \texttt{doctor}. Sie unterstützen Durchsatz-/Latenzschätzung, Interface-Inspektion und Grunddiagnostik.

\section{Überblick über das Gesamtsystem}
\subsection{Anforderungen an den Algorithmus}
Die vom Ouster~OS1 erfassten Sensordaten bilden die Grundlage für eine Messkette, die relevante Objekte des Umfelds erkennt und strukturiert ausgibt. Ziel ist eine Objektliste mit Position, Orientierung und Abmessungen.

Für eine zuverlässige Objekterkennung ist eine geeignete Aufbereitung der Rohdaten erforderlich. Bodenpunkte und Störsignale sind zu entfernen; die Punktdichte ist zu reduzieren, ohne die Geometrie relevanter Objekte wesentlich zu verfälschen. Die Parameter der Schritte müssen anpassbar bleiben, um unterschiedliche Szenarien abbilden zu können.

Die Verarbeitung erfolgt unter Echtzeitbedingungen: Alle Prozesse sind innerhalb einer festen Zeitspanne abzuschließen, damit Ergebnisse kontinuierlich bereitgestellt werden (Ziel-Gesamtlatenz \(<100\,\text{ms}\) bei \(10\,\text{Hz}\)). Ein ressourcenschonender Betrieb auf Standardhardware ist sicherzustellen. Ergebnisse werden maschinenlesbar und zur schnellen Kontrolle im Terminal ausgegeben; zudem ist die Integration in die bestehende GUI vorgesehen.

Die wichtigsten Anforderungen an die entwickelte Messkette sind in Tabelle~\ref{tab:anforderungen_messkette} zusammengefasst.

\begin{table}[h!]
  \centering
  \begin{tabular}{|c|p{12cm}|}
    \hline
    \textbf{Nr.} & \textbf{Anforderung} \\ \hline
    1 & Eingabe: Sensordaten des Ouster~OS1-LiDAR \\ \hline
    2 & Ausgabe: Objektliste mit Position, Orientierung und Abmessungen \\ \hline
    3 & Entfernung von Bodenpunkten und Störsignalen \\ \hline
    4 & Reduzierung der Punktwolkendichte zur Rechenoptimierung \\ \hline
    5 & Anpassbare Parameter für unterschiedliche Umgebungen \\ \hline
    6 & Echtzeitverarbeitung: Latenz \textless{}100\,ms bei 10\,Hz Sensorrate \\ \hline
    7 & Ressourcenschonender Betrieb auf Standardhardware \\ \hline
    8 & Maschinenlesbare und terminalbasierte Ergebnisausgabe \\ \hline
    9 & Integration in die bestehende grafische Benutzeroberfläche (GUI) \\ \hline
  \end{tabular}
  \caption{Anforderungen an die Messkette zur Umfelderkennung}
  \label{tab:anforderungen_messkette}
\end{table}

\subsection{Datenerfassung}
Die Datenerfassung umfasst Aufnahme und Übertragung der OS1-Punktwolken. Der grundlegende Aufbau orientiert sich an \parencite{OusterOS1}. Die Daten werden über eine Ethernet-Verbindung an den Embedded-PC (dSPACE MAB~III) übertragen; zur Anbindung kommt ein USB-A-auf-Ethernet-Adapter des Typs \emph{Renkforce RF-4708614} mit einer Übertragungsrate bis \(1\,\text{Gbit/s}\) zum Einsatz (vgl. \cite{conrad2025}). Die Kommunikation zwischen Sensor und Embedded-System erfolgt über ROS~2.

Der Ouster~OS1 publiziert seine Messdaten als \texttt{sensor\_msgs/PointCloud2} auf \texttt{/ouster/points}. Die Punktwolken enthalten pro Punkt Kartesierkoordinaten \((x,y,z)\), die Intensität des reflektierten Signals sowie den Ring-Index des jeweiligen Lasers; der Zeitstempel wird direkt aus der LiDAR-Firmware übernommen. Die Punktwolken werden über das lokale ROS~2-Netz an den Laptop übertragen, auf dem sie wahlweise in \emph{RViz2} visualisiert oder über die GUI verarbeitet werden können.

\begin{figure}[h!]
  \centering
  \includegraphics[width=0.85\textwidth]{Bilder/messkette_sensordaten.png}
  \caption{Aufbau der Datenerfassung mit dem Ouster~OS1-LiDAR (überarbeitet nach \parencite{OusterOS1}).}
  \label{fig:messkette_sensordaten}
\end{figure}

\subsection{Pipeline (Umsetzung)}
Abbildung~\ref{fig:ros2-pipeline-alltopics} zeigt die ROS~2-Pipeline von der Aufnahme über Vorverarbeitung, Bodensegmentierung und Cluster/Bounding-Boxes bis zur Objektverfolgung. Jeder Knoten hat klar definierte Ein-/Ausgänge und Topics.

\subsection*{Knotenübersicht (Ein-/Ausgänge und Parameter)}
\begin{table}[h!]
  \centering
  \begin{tabular}{|l|l|l|l|}
    \hline
    \textbf{Node} & \textbf{Eingang (Typ)} & \textbf{Ausgang (Typ)} & \textbf{Schlüsselparameter} \\ \hline
    Input     & /ouster/points (PointCloud2) & /points\_cropped (PointCloud2) & ROI (min/max) \\ \hline
    CropBox   & /points\_cropped (PointCloud2) & /points\_voxel (PointCloud2) & Box, keep\_organized \\ \hline
    VoxelGrid & /points\_voxel (PointCloud2) & /obstacle\_points (PointCloud2) & voxel\_size \\ \hline
    GroundSeg & /obstacle\_points (PointCloud2) & /detections\_raw (custom) & max\_plane\_dist, iters \\ \hline
    Clustering& /detections\_raw (custom) & /tracks\_raw (custom) & cluster\_tol, min/max\_size \\ \hline
    Tracking  & /tracks\_raw (custom) & /tracks\_markers (viz) & dt, gating \\ \hline
  \end{tabular}
  \caption{Knoten, I/O-Themen und exemplarische Parameter der Pipeline.}
  \label{tab:nodes-io}
\end{table}

\begin{figure}
\centering
\begin{tikzpicture}[
    node distance=2.2cm,
    process/.style={rectangle, rounded corners, draw=black, fill=blue!5,
                    text centered, minimum width=4.2cm, minimum height=1cm},
    startstop/.style={ellipse, draw=black, fill=gray!15,
                      text centered, minimum width=3cm, minimum height=1cm},
    rviz/.style={rectangle, draw=black, fill=green!10, rounded corners,
                 text centered, minimum width=3.5cm, minimum height=0.9cm,
                 font=\scriptsize},
    arrow/.style={thick,->,>=stealth},
    every node/.style={font=\small}
  ]

  % Hauptlinie
  \node (start)   [startstop] {Start};
  \node (input)   [process, below of=start] {Messdaten einlesen};
  \node (crop)    [process, below of=input] {CropBox-Filter};
  \node (voxel)   [process, below of=crop] {VoxelGrid-Filter};
  \node (ground)  [process, below of=voxel] {Bodensegmentierung};
  \node (cluster) [process, below of=ground] {Cluster-Extraktion \& Bounding Boxes};
  \node (track)   [process, below of=cluster] {Objektverfolgung};
  \node (end)     [startstop, below of=track] {Ende};

  % RViz2-Knoten (rechts)
  \node (rviz_input)  [rviz, right=4.1cm of input]  {RViz2};
  \node (rviz_crop)   [rviz, right=4.1cm of crop]   {RViz2};
  \node (rviz_voxel)  [rviz, right=4.1cm of voxel]  {RViz2};
  \node (rviz_ground) [rviz, right=4.1cm of ground] {RViz2};
  \node (rviz_cluster)[rviz, right=2.8cm of cluster]{RViz2};
  \node (rviz_track)  [rviz, right=4.1cm of track]  {RViz2};

  % Vertikale Pfeile (Pipeline)
  \draw[arrow] (start) -- (input);
  \draw[arrow] (input) -- node[right]{\scriptsize \texttt{/ouster/points}} (crop);
  \draw[arrow] (crop)  -- node[right]{\scriptsize \texttt{/points\_cropped}} (voxel);
  \draw[arrow] (voxel) -- node[right]{\scriptsize \texttt{/points\_voxel}} (ground);
  \draw[arrow] (ground)-- node[right]{\scriptsize \texttt{/obstacle\_points}} (cluster);
  \draw[arrow] (cluster)-- node[right]{\scriptsize \texttt{/detections\_raw}} (track);
  \draw[arrow] (track) -- node[right]{\scriptsize \texttt{/tracks\_raw}} (end);

  % Pfeile zu RViz2 (alle Schritte)
  \draw[arrow] (input.east)  -- node[above,sloped]{\scriptsize \texttt{/ouster/points}} (rviz_input.west);
  \draw[arrow] (crop.east)   -- node[above,sloped]{\scriptsize \texttt{/points\_cropped}} (rviz_crop.west);
  \draw[arrow] (voxel.east)  -- node[above,sloped]{\scriptsize \texttt{/points\_voxel}} (rviz_voxel.west);
  \draw[arrow] (ground.east) -- node[above,sloped]{\scriptsize \texttt{/obstacle\_points}} (rviz_ground.west);
  \draw[arrow] (cluster.east)-- node[above,sloped]{\scriptsize \texttt{/detections\_markers}} (rviz_cluster.west);
  \draw[arrow] (track.east)  -- node[above,sloped]{\scriptsize \texttt{/tracks\_markers}} (rviz_track.west);

  % Caption & Label
  \end{tikzpicture}
  \caption{ROS~2-Algorithmus mit Nodes, Topic-Verbindungen und RViz2-Ausgaben.}
  \label{fig:ros2-pipeline-alltopics}
\end{figure}

\section{Deployment und Packaging}
Die Umsetzung erfolgt im ROS~2-Workspace (\texttt{colcon}); \texttt{src/} enthält die Pakete, \texttt{build/}/\texttt{install/}/\texttt{log/} werden erzeugt. Abhängigkeiten werden in \texttt{package.xml} deklariert und in \texttt{CMakeLists.txt} gebunden. Eine Trennung in Bibliotheken (Kernlogik) und Executables (ROS~2-Anbindung) erhöht die Wiederverwendbarkeit.

\subsection*{Launch, Parameter und Namensräume}
Start und Parametrisierung erfolgen über Launch-Dateien und \texttt{.yaml}-Parameter (z.\,B. Voxelgröße, Filterradien, QoS-Tiefe). Namensräume strukturieren Topics (z.\,B. Sensorpfad, Verarbeitungspfad) und erlauben parallele Instanziierung. Visualisierung: konsequent \textbf{RViz2}.

\subsection*{Build und Start (Beispiel)}
\begin{verbatim}
colcon build --symlink-install --cmake-args -DCMAKE_BUILD_TYPE=Release
source install/setup.bash
ros2 launch my_pipeline bringup.launch.py use_sim_time:=false
\end{verbatim}

\subsection*{Minimalbeispiele: package.xml und CMake}
\begin{verbatim}
<!-- package.xml (Ausschnitt) -->
<package format="3">
  <name>my_pipeline</name>
  <version>0.1.0</version>
  <buildtool_depend>ament_cmake</buildtool_depend>
  <depend>rclcpp</depend>
  <depend>sensor_msgs</depend>
  <depend>pcl_conversions</depend>
</package>

# CMakeLists.txt (Ausschnitt)
find_package(ament_cmake REQUIRED)
find_package(rclcpp REQUIRED)
find_package(sensor_msgs REQUIRED)
find_package(PCL REQUIRED)
add_executable(voxel_node src/voxel_node.cpp)
ament_target_dependencies(voxel_node rclcpp sensor_msgs)
target_link_libraries(voxel_node ${PCL_LIBRARIES})
install(TARGETS voxel_node DESTINATION lib/${PROJECT_NAME})
\end{verbatim}

\section{Leistungsmetriken und Verifikation}
\subsection*{E2E-Latenz}
Ziel: \(<\,100\,\text{ms}\) bei \(10\,\text{Hz}\). Methodik: Zeitstempel an Modulgrenzen (Eingang/Ausgang) und Auswertung von Mittelwert, P95, P99 über lange Sequenzen; Messung im Live-Betrieb und bei Wiedergabe identischer Daten aus \texttt{rosbag2}. Instrumentierung: Eingangszeit aus \texttt{header.stamp}; Zwischenzeiten in Modulen erfassen/mitführen. Gemeinsame Zeitbasis (z.\,B. deaktivierte Simulationszeit, externe Synchronisation) wird vorausgesetzt.

\subsection*{Ressourcennutzung (CPU/RAM)}
Messung pro Knoten unter repräsentativer Last. Optimierungen (Datenlayout, Kopiervermeidung, Vorallokation) werden dokumentiert und quantifiziert. Hilfsmittel: \texttt{ros2 topic hz/bw/delay}, Systemprofiler (\texttt{top}/\texttt{htop}), persistente Artefakte (CSV/JSON) für Variantenvergleiche.

\subsection*{Trace-gestützte Analyse und Fehlermodi}
Feingranulare Messungen mit Tracing (z.\,B. \emph{ros2\_tracing}) erfassen Callback-Laufzeiten, Scheduling-Lücken und E2E-Latenzen. Häufige Fehlermodi: überfüllte Queues (zu kleine Tiefe), unpassende QoS-Mischungen zwischen Verarbeitung/Visualisierung, fehlendes \emph{intra-process}, unsaubere Zeitbasen. Gegenmaßnahmen sind in den jeweiligen Abschnitten benannt.


\printbibliography

\end{document}
