% ====== 8 Zusammenfassung und Ausblick ======
\chapter{Zusammenfassung und Ausblick}
\section{Zusammenfassung}
Die vorliegende Arbeit entwickelte eine \ac{LiDAR}-basierte Verarbeitungskette, die alle Schritte von der Vorverarbeitung bis zur
Objektverfolgung integriert. Ausgehend von einer \ac{ROS}~2-basierten Systemarchitektur wurden zunächst robuste Verfahren zur
Normalisierung und Filterung der Punktwolken aufgebaut, die eine verlässliche Grundlage für nachgelagerte Algorithmen schaffen.
Darauf aufbauend ermöglicht die Bodensegmentierung eine stabile Trennung zwischen Fahrbahn- und Objektpunkten, während die
Clustering-Verfahren mit datengetriebenen Bounding-Boxen eine kompakte und interpretierbare Objektrepräsentation liefern.
Die anschließende Verfolgung stellt durch geeignete Datenassoziation und modellbasierte Prädiktion sicher, dass auch bei
temporären Sensorausfällen oder Abschattungen konsistente Trajektorien berechnet werden. In umfangreichen Versuchsfahrten
bestätigte die Verarbeitungskette ihre Echtzeitfähigkeit: Die Verarbeitungslatenzen blieben im typischen Fahrbetrieb deutlich unter
der Eingangstaktung des Sensors, sodass ausreichend Reserven für zusätzliche Funktionen bestehen. Insgesamt zeigt die Arbeit,
dass die Kombination aus klassischen Verfahren und sorgfältiger Parametrierung eine zuverlässige 360°-Umfelderfassung mit
hochaufgelösten \ac{LiDAR}-Daten ermöglicht.
\section{Ausblick}
Die erzielten Ergebnisse legen mehrere Ansatzpunkte für weiterführende Arbeiten nahe. Kurzfristig bietet sich eine adaptive
Parametrierung der Filter- und Segmentierungsstufen an, um die Verarbeitungskette dynamisch an unterschiedliche Wetter- oder
Verkehrssituationen anzupassen. Perspektivisch könnten probabilistische Filter oder graphbasierte Optimierungsverfahren die
Tracking-Genauigkeit in komplexen Szenen erhöhen und robuste Reidentifikation ermöglichen. Darüber hinaus lässt sich durch
Sensorfusion mit Kamera- oder Radardaten sowohl die semantische als auch die geometrische Konsistenz steigern, etwa über
gemeinsame Objektmerkmale oder \ac{BEV}-Repräsentationen. Ergänzend sollte geprüft werden, wie \ac{SLAM}-Verfahren auf Basis der
\ac{LiDAR}-Daten eine lokale Karte aufbauen und die Eigenbewegung präzisieren können, um die Objekttrajektorien in ein konsistentes
Weltkoordinatensystem einzubetten. Für eine spätere Integration in hochautomatisierte Fahrfunktionen sind zusätzliche
Safety-Mechanismen wie Zustandsüberwachung, Plausibilitätsprüfungen und Fallback-Strategien wünschenswert. Eine kontinuierliche
Evaluation mit erweiterten Datensätzen und realen Fahrversuchen wird entscheidend sein, um die Übertragbarkeit auf
unterschiedliche Fahrzeugplattformen und Einsatzgebiete zu sichern und das volle Potenzial der \ac{LiDAR}-basierten
Umfelderfassung auszuschöpfen.