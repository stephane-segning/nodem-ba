% ====== Kurzfassung / Zusammenfassung ======
\chapter*{Kurzfassung}
\addcontentsline{toc}{chapter}{Kurzfassung}
\noindent
Die vorliegende Arbeit entwickelt eine modulare ROS2-Verarbeitungskette für die \ac{LiDAR}-basierte Umfelderkennung des Projektautomobils \textit{CarCeptionX}. Aufbauend auf der Ouster-\ac{OS1}-Messkette werden die Stufen Zuschnitt (CropBox), Voxel-Filter, RANSAC-Bodensegmentierung, Cluster-/Bounding-Box-Bildung und ein kalmanfilterbasiertes Tracking zu einer durchgängigen Pipeline verknüpft. Ziel ist eine echtzeitfähige Erkennung von Verkehrsteilnehmern bei mindestens \SI{90}{\percent} Detektionsrate.

Eine Parameterstudie zu Voxelgröße, RANSAC-Distanzschwelle und Cluster-Toleranz zeigt, dass moderate Einstellungen die robustesten Ergebnisse liefern: \(\textit{voxel\_size} \approx 0{,}15{-}0{,}20\,\text{m}\), \(\textit{distance\_threshold} \approx 0{,}15\,\text{m}\) und \(\textit{cluster\_tolerance} \approx 0{,}50\,\text{m}\) vermeiden sowohl Übersegmentierung als auch das Verschmelzen benachbarter Objekte. Damit erreicht die Pipeline in unterschiedlichen Szenarien Detektionszahlen, die eng an den realen Objektzahlen liegen.

Fahrtests auf gerader Strecke, in Kurven und auf unebenen Stadtabschnitten bestätigen die Echtzeitfähigkeit: Die End-to-End-Latenz von zugeschnittenen Punktwolken bis zu den Tracks liegt je nach Szenario zwischen rund \SI{20}{\milli\second} und \SI{86}{\milli\second} bei CPU-Lasten unter \SI{35}{\percent} und einem Speicherbedarf um \SI{200}{\mega\byte}. Gleichzeitig bleibt die Ausgaberate der Tracks im Bereich von \SIrange{1.8}{3.0}{\hertz}; nur in statischen Szenen treten kurzlebige Track-IDs auf, die auf weiteres Feintuning des Track-Managements hindeuten, die Kernanforderungen aber nicht beeinträchtigen.

In realen Messszenarien auf urbanen Teststrecken zeigt die Verarbeitungskette stabile Resultate: Die Bodensegmentierung trennt befahrbare Fläche und Hindernisse verlässlich, die Clusterbildung erzeugt weitgehend trennscharfe Objektboxen, und das Tracking liefert kohärente Trajektorien auch bei kurzzeitigen Abschattungen. Die gemessenen Verarbeitungsraten liegen je nach Szene bei \SIrange{2.5}{3.7}{\hertz} für die Punktwolkenstufen und \SIrange{1.8}{3.0}{\hertz} für die Track-Ausgabe, womit die geforderten Echtzeitanforderungen im Testaufbau eingehalten werden. Die Arbeit schließt mit einer Bewertung der erzielten Leistung und einem Ausblick auf Erweiterungen wie Sensorfusion, semantische Segmentierung und eine engere Integration in das Fahrzeugsystem.

\cleardoublepage