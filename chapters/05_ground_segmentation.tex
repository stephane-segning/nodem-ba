\chapter{Bodensegmentierung}
\label{chap:bodensegmentierung}

Die Bodensegmentierung ist ein essenzieller Vorverarbeitungsschritt innerhalb der Verarbeitungskette zur Objekterkennung.
Ihr Ziel besteht darin, Bodenpunkte zuverlässig von Hindernissen und anderen Objekten zu trennen, um die nachfolgenden Schritte – insbesondere die Cluster- und Objekterkennung – zu erleichtern.
Dieses Kapitel führt die Anforderungen an eine robuste Bodensegmentierung ein, vergleicht gängige Verfahren anhand klarer Bewertungskriterien und stellt die implementierte \ac{RANSAC}-Variante im Detail vor. Dadurch ergibt sich ein roter Faden von den Anforderungen über den Methodenvergleich bis hin zur konkreten Umsetzung.


\section{Anforderungen an die Bodensegmentierung}
\label{sec:anforderungen_bodenseg}

Die wichtigsten Anforderungen an ein Bodensegmentierungsverfahren lassen sich nach~\cite{gomes2023survey} in mehreren Kriterien zusammenfassen, die in Tabelle~\ref{tab:anforderungen_bodensegmentierung} dargestellt sind.

\begin{table}[H]
\centering
\renewcommand{\arraystretch}{1.3} % augmente l'espacement vertical
\setlength{\tabcolsep}{8pt}       % ajuste l'espacement horizontal
\begin{tabular}{|p{4cm}|p{9.5cm}|}
\hline
\textbf{Kriterium} & \textbf{Beschreibung} \\ \hline

\textbf{Echtzeitfähigkeit} & 
Das Verfahren muss Punktwolken in Echtzeit verarbeiten können (z.\,B. <100\,ms pro Frame), 
um eine kontinuierliche Fahrzeugsteuerung zu ermöglichen. \\ \hline

\textbf{Rechenaufwand} & 
Geringe Rechen- und Speicheranforderungen sind notwendig, 
da die Algorithmen häufig auf eingebetteten Systemen mit begrenzten Ressourcen laufen. \\ \hline

\textbf{Segmentierungs\-robustheit} & 
Das Verfahren sollte robust gegenüber Über- und Untersegmentierung sein 
und Bodenpunkte korrekt von Hindernissen trennen. \\ \hline

\textbf{Leistung bei steigenden Hindernissen} & 
Die Methode sollte sanft ansteigende Strukturen (z.\,B. Rampen oder Bordsteine) 
korrekt als Teil des Bodens erkennen. \\ \hline

\textbf{Leistung bei unebenem Gelände} & 
Auch bei Neigungen oder unregelmäßigen Bodenoberflächen 
muss die Bodenschätzung stabil bleiben. \\ \hline

\textbf{Leistung bei spärlichen Daten} & 
Das Verfahren sollte bei geringer Punktdichte (z.\,B. großer Abstand zum Sensor) 
zuverlässige Ergebnisse liefern. \\ \hline
\end{tabular}

\vspace{5pt}
\caption{Hauptanforderungen an die Bodensegmentierung (nach~\cite{gomes2023survey})}
\label{tab:anforderungen_bodensegmentierung}
\end{table}

Aus den sechs Anforderungen werden für die weitere Analyse vier Bewertungskriterien abgeleitet:

\begin{table}[H]
\centering
\renewcommand{\arraystretch}{1.2}
\begin{tabular}{|c|p{4cm}|p{9cm}|}
\hline
\textbf{Nr.} & \textbf{Kriterium} & \textbf{Beschreibung} \\ \hline
1 & Echtzeitfähigkeit & Bewertet, ob die Methode unter den gegebenen Rechenbedingungen (10\,Hz Sensorrate) eine kontinuierliche Verarbeitung der Punktwolke in Echtzeit ermöglicht. \\ \hline
2 & Robustheit & Misst die Widerstandsfähigkeit gegenüber Rauschen, Ausreißern und unterschiedlichen Geländetypen (z.\,B. unebenes Terrain oder variable Punktdichten). \\ \hline
3 & Genauigkeit & Beschreibt die Fähigkeit der Methode, die Bodenpunkte präzise von Nicht-Bodenpunkten zu trennen und somit eine zuverlässige Segmentierung zu gewährleisten. \\ \hline
4 & Rechenaufwand & Bewertet den benötigten Ressourcenverbrauch (CPU/GPU-Zeit und Speicherbedarf) für die Durchführung der Bodensegmentierung. \\ \hline
\end{tabular}
\caption{Bewertungskriterien zur Analyse der Bodensegmentierungsmethoden.}
\label{tab:bewertungskriterien}
\end{table}

Die Auswahl dieser vier Bewertungskriterien orientiert sich an den Einsatzbedingungen urbaner \ac{LiDAR}-Szenarien: komplexe Geometrien und bewegte Objekte erfordern Robustheit und Genauigkeit, das Bordzeitbudget verlangt Echtzeitfähigkeit, und eingeschränkte Ressourcen machen einen geringen Rechenaufwand nötig. Tabelle~\ref{tab:bewertungskriterien} fasst die Verdichtung der sechs Anforderungen aus Tabelle~\ref{tab:anforderungen_bodensegmentierung} zusammen, indem inhaltlich verwandte Punkte gebündelt werden:
\begin{itemize}
    \item \textbf{Robustheit} fasst die Leistungsbewertung bei steigenden Hindernissen, unebenem Gelände und spärlichen Daten zusammen,
    weil alle drei Aspekte die Widerstandsfähigkeit gegenüber unterschiedlichen Gelände- und Sensordichtevariationen abprüfen.
    \item \textbf{Genauigkeit} nimmt den Aspekt der korrekten Trennung von Boden- und Nicht-Bodenpunkten aus der Segmentierungsrobustheit auf,
    um die resultierende Segmentierungsqualität klar zu bewerten.
    \item \textbf{Echtzeitfähigkeit} und \textbf{Rechenaufwand} bleiben als eigenständige Kriterien erhalten,
    da sie die Laufzeit- und Ressourcenanforderungen unabhängig von den Szenarioeigenschaften adressieren.
\end{itemize}

\section{Methoden der Bodensegmentierung}
\label{chap:bodenmethoden}

Nach~\cite{gomes2023survey} lassen sich die gängigen Verfahren zur Bodensegmentierung
in fünf Hauptkategorien einteilen (siehe Abbildung~\ref{fig:boden_taxonomie}):

\begin{itemize}
    \item 2.5D-Gitterbasierte Verfahren
    \item Bodenmodellierung (Ground Modelling)
    \item Methoden auf Basis benachbarter Punkte und lokaler Merkmale
    \item Verfahren höherer Ordnung (Higher-Order Inference)
    \item Lernbasierte Verfahren (Deep Learning)
\end{itemize}

\begin{figure}[H]
  \centering
  \includegraphics[width=0.9\textwidth]{Bilder/boden_taxonomie.png}
  \caption{Klassifizierung und Systematisierung bestehender Bodensegmentierungsmethoden}
  \label{fig:boden_taxonomie}
\end{figure}


Diese Klassifikation deckt sowohl klassische geometrische Ansätze als auch moderne, neuronale Verfahren ab. 
Im Folgenden werden die wichtigsten Prinzipien und repräsentativen Algorithmen jeder Kategorie erläutert.

\subsection{2.5D-Gitterbasierte Verfahren}
% ============================

Ein verbreiteter Ansatz zur Bodensegmentierung besteht darin, die dreidimensionale Punktwolke in eine zweidimensionale Rasterdarstellung zu überführen.  
Dabei werden die Punkte nach ihren Koordinaten in diskrete Zellen eingeteilt, sodass jede Zelle statistische Höheninformationen über die in ihr enthaltenen Punkte speichert.  
Dieser sogenannte 2.5D-Ansatz reduziert die Komplexität der Verarbeitung erheblich, da die Analyse nicht mehr im vollen 3D-Raum, sondern auf einer strukturierten Gitterebene erfolgt.

Douillard et al.~\cite{douillard2011segmentation} präsentieren ein solches Verfahren auf Basis von \textit{Elevation Maps}, bei dem jede Gitterzelle durch den Mittelwert der Höhenwerte ihrer Punkte beschrieben wird.  
Zellen mit geringer Höhenvarianz werden als Boden klassifiziert, während größere Abweichungen auf Objekte oder Hindernisse hinweisen.  
Durch diese Reduktion auf lokale Höhenstatistiken kann die Methode Bodenflächen effizient und robust in urbanen Umgebungen identifizieren, ohne den gesamten Punktwolkenraum verarbeiten zu müssen.


% ============================
\subsection{Bodenmodellierung (Ground Modelling)}
% ============================

Diese Methoden approximieren die Bodenfläche durch mathematische Modelle, typischerweise in Form von Ebenen oder Linien, um die Trennung zwischen Boden- und Nichtbodenpunkten zu ermöglichen.

\textbf{Plane Fitting:}  
In \cite{hu2013robust} wird die Identifikation von Bodenpunkten mithilfe der \textit{Random Sample Consensus (\ac{RANSAC})}-Methode beschrieben, 
bei der eine Ebene an die niedrigsten Punkte angepasst und Punkte mit geringem orthogonalen Abstand als \textit{Inlier} klassifiziert werden.  
Ein ähnlicher Ansatz teilt die Punktwolke in konzentrische Zonen auf und passt für jede Zone lokal angepasste Ebenen an, 
wobei die \textit{Principal Component Analysis (PCA)} eingesetzt wird, um die Hauptrichtung der Punktverteilung zu bestimmen und daraus die bestmögliche Ebenenorientierung zu berechnen \cite{lim2020fast}.  
Dieser Ansatz erreicht eine hohe Präzision (F1-Score $\approx 0.93$) bei gleichzeitigem Echtzeitverhalten.

\textbf{Linienextraktion:}
Ein auf lokalen Linienanpassungen basierendes Verfahren modelliert die entlang eines Laserstrahls erfassten Punkte als nahezu linearen Verlauf \cite{himmelsbach2010fast}.
Dieses Verfahren ist recheneffizient und eignet sich für Echtzeitverarbeitung, weist jedoch Schwächen in stark unebenem oder komplexem Gelände auf.

\textbf{Gaussian Process Regression (GPR):}  
Die \textit{Gaussian Process Regression (GPR)} wurde zur Modellierung der Bodenhöhe eingeführt, 
wobei der Höhenverlauf als kontinuierliche Funktion mit zugehöriger Unsicherheitsabschätzung beschrieben wird \cite{douillard2011segmentation}.  
Durch die Verwendung nichtstationärer Kovarianzfunktionen kann sich das Modell lokal an unterschiedliche Geländestrukturen anpassen und dadurch eine hohe Genauigkeit auch in unregelmäßigem Terrain erreichen \cite{chen2014real}.

% ============================
\subsection{Benachbarte Punkte und lokale Merkmale}
% ============================

Diese Algorithmen analysieren geometrische Beziehungen zwischen benachbarten Punkten in der Punktwolke.  
Dabei wird häufig die vertikale Kanalstruktur moderner \ac{LiDAR}-Sensoren (z.\,B. Velodyne VLP-16 oder Ouster OS1-64) ausgenutzt, um lokale Abhängigkeiten entlang der Scanlinien zu erkennen.

\textbf{Kanalbasierte Verfahren:}  
In \cite{chu2019ground} wird die Bodenextraktion entlang vertikaler Scans beschrieben, 
bei der lokale Höhenunterschiede und Gradienten ausgewertet werden.  
Punkte zwischen einem Startbodenpunkt und einem definierten Schwellenwert werden als Boden klassifiziert.  
Das Verfahren ist recheneffizient, reagiert jedoch empfindlich auf eine ungenaue Parametrierung, etwa bei der Wahl des Höhen- oder Gradientschwellenwerts.

\textbf{Region-Growing und Clustering:}  
In \cite{moosmann2009segmentation} wird ein graphenbasiertes Region-Growing-Verfahren vorgestellt, 
bei dem benachbarte Punkte iterativ zu Regionen zusammengefügt werden, 
sofern lokale geometrische Kriterien (z.\,B. Konvexität) erfüllt sind.  
Ein alternativer Ansatz kombiniert voxelbasiertes Clustering mit statistischer Analyse, 
um Bodencluster zuverlässig zu isolieren \cite{douillard2011segmentation}.  
Solche Methoden liefern stabile Ergebnisse auch in komplexen Szenen, 
sind jedoch rechenintensiver und daher weniger für Echtzeitanwendungen geeignet.

\textbf{Range-Image-Methoden:}  
In \cite{bogoslavskyi2016fast} wird die Punktwolke in ein zweidimensionales Entfernungsbild (\textit{Range Image}) projiziert, 
bei dem jeder Pixel den Abstand eines Messpunkts zum Sensor repräsentiert.  
Diese Repräsentation ermöglicht eine effiziente Definition von Nachbarschaften und vereinfacht die anschließende Segmentierung erheblich.  
Mit diesem Ansatz kann die Bodenextraktion in wenigen Millisekunden pro Frame durchgeführt werden, was eine Echtzeitverarbeitung erlaubt.

% ============================
\subsection{Verfahren höherer Ordnung}
% ============================

Ansätze dieser Kategorie verwenden probabilistische Graphmodelle wie 
\textit{Markov Random Fields (MRF)} oder \textit{Conditional Random Fields (CRF)}, 
um Abhängigkeiten zwischen benachbarten Punkten explizit zu modellieren.  
Durch die Berücksichtigung solcher räumlicher Korrelationen können Fehlklassifikationen, 
insbesondere in spärlichen oder verrauschten Punktwolken, deutlich reduziert werden.

In \cite{guo2011ground} wird ein MRF-Modell mit dem \textit{Belief Propagation (BP)}-Verfahren kombiniert, 
das die Wahrscheinlichkeiten einzelner Punktzuordnungen iterativ aktualisiert, 
um eine konsistente Bodenfläche auch in unebenem Gelände zu rekonstruieren.  
Ein weiterentwickelter Ansatz integriert zeitliche Abhängigkeiten in ein CRF-Modell, 
wodurch die Konsistenz zwischen aufeinanderfolgenden Frames verbessert 
und die Stabilität der Segmentierung bei Bewegungen erhöht wird \cite{rummelhard2015temporal}.

% ============================
\subsection{Lernbasierte Verfahren}
% ============================

In den letzten Jahren haben sich tief neuronale Netze als besonders leistungsfähige Ansätze für die Punktwolkensegmentierung etabliert.  
Je nach Architektur werden die Sensordaten entweder direkt in Punktform verarbeitet oder zuvor in strukturierte Darstellungen überführt, um eine effiziente Merkmalsextraktion zu ermöglichen.

\textbf{PointNet- und Voxel-basierte Modelle:}  
Das in \cite{qi2017pointnet} vorgestellte PointNet-Framework ermöglicht die direkte Verarbeitung unstrukturierter Punktwolken, indem es für jeden Punkt Merkmale extrahiert und diese global aggregiert.  
Zur Erfassung lokaler geometrischer Abhängigkeiten wurde das Konzept in PointNet++ erweitert.  
Alternativ unterteilen voxelbasierte Verfahren wie VoxelNet \cite{zhou2018voxelnet} oder PointPillars \cite{lang2019pointpillars} die Punktwolke in diskrete 3D-Zellen (Voxel) bzw. Säulen und verwenden dreidimensionale Faltungsnetzwerke (3D-CNNs) zur Merkmalsanalyse.  

\textbf{Bildbasierte Ansätze:}  
In \cite{wu2018squeezeseg} und \cite{milioto2019rangenet++} werden Punktwolken in zweidimensionale Entfernungsbilder (\textit{Range Images}) projiziert, sodass konventionelle 2D-Faltungsnetzwerke auf \ac{LiDAR}-Daten angewendet werden können.  
Diese Repräsentation ermöglicht eine Echtzeitverarbeitung auf GPUs bei gleichzeitig hoher Segmentierungsgenauigkeit.  

\textbf{Spezialisierte Netze für Bodensegmentierung:}  
Ein speziell für die Bodenerkennung entwickeltes neuronales Modell ist GndNet \cite{paigwar2020gndnet}.  
Das Netz basiert auf einem zweidimensionalen Gittermodell, in dem für jede Zelle die Bodenhöhe vorhergesagt wird.  
Dieses Verfahren erreicht eine mittlere Intersection-over-Union (\ac{IoU}) von 83,6\,\% bei einer Laufzeit von nur 17,9\,ms pro Frame und ist somit für Echtzeitanwendungen geeignet.

% ============================
\subsection{Zusammenfassung}
% ============================

Tabelle~\ref{tab:vergleich_bodenmethoden} fasst die wesentlichen Eigenschaften der vorgestellten Bodensegmentierungsmethoden zusammen.
Sie verdeutlicht die jeweiligen Stärken und Grenzen der Ansätze sowie ihre typischen Einsatzgebiete in der mobilen Robotik und Umfelderkennung.

\begin{table}[H]
\centering
\renewcommand{\arraystretch}{1.2}
\begin{tabular}{|p{3.5cm}|p{4cm}|p{4cm}|p{3cm}|}
\hline
\textbf{Methodenkategorie} & \textbf{Vorteile} & \textbf{Nachteile} & \textbf{Typisches Einsatzgebiet} \\ \hline
2.5D-Gitterbasierte Verfahren & Geringe Rechenlast, robust auf ebenem Gelände & Begrenzte Genauigkeit bei Überhängen oder Brücken & Stadt- und Straßenszenarien \\ \hline
Bodenmodellierung & Hohe Präzision, einfache mathematische Umsetzung & Geringe Robustheit bei komplexen Oberflächenformen & Flaches bis leicht geneigtes Terrain \\ \hline
Lokale Merkmalsanalyse & Unempfindlich gegenüber Dichteänderungen & Starke Abhängigkeit von Parameterwahl und Sensorgeometrie & Dynamische oder unstrukturierte Umgebungen \\ \hline
Probabilistische Graphmodelle (MRF/CRF) & Hohe Konsistenz durch Nachbarschaftsbeziehungen & Hohe Rechenkomplexität, geringe Echtzeitfähigkeit & Forschungs- und Entwicklungsumgebungen \\ \hline
Lernbasierte Verfahren & Sehr hohe Genauigkeit, anpassungsfähig durch Training & Großer Trainings- und Hardwareaufwand & Autonomes Fahren, Echtzeit-Perzeption \\ \hline
\end{tabular}
\caption{Vergleich der Bodensegmentierungsmethoden in Bezug auf Rechenaufwand, Robustheit und Einsatzgebiet (nach~\cite{gomes2023survey}).}
\label{tab:vergleich_bodenmethoden}
\end{table}

\begin{table}[h!]
  \centering
  \renewcommand{\arraystretch}{1.2}
  \resizebox{\textwidth}{!}{%
    \begin{tabular}{|l|c|c|c|c|}
      \hline
      \textbf{Methode} & \textbf{Echtzeitfähigkeit} & \textbf{Robustheit} & \textbf{Genauigkeit} & \textbf{Rechenaufwand} \\
      \hline
      2.5D-Gitterbasierte Verfahren & Hoch (Zellstatistiken pro Raster) & Mittel (stabil bei moderaten Höhenänderungen) & Mittel (verlustbehaftete 2.5D-Projektion) & Gering \\ \hline
      \ac{RANSAC}-Ebenenanpassung (Plane Fitting) & Mittel (Iterationsbudget begrenzt Laufzeit) & Hoch (ausreißerresistent) & Hoch (präzise Ebenenmodellierung) & Mittel bis hoch \\ \hline
      Linienextraktion entlang Scanlinien & Hoch (lineare Fits pro Strahl) & Niedrig bis mittel (empfindlich auf unebenes Gelände) & Mittel & Sehr gering \\ \hline
      Gaussian Process Regression (GPR) & Niedrig (aufwändige Kernelberechnung) & Hoch (passt sich lokalem Terrain an) & Hoch & Hoch \\ \hline
      Kanalbasierte Verfahren (gradientenbasiert) & Hoch (schlanke Schwellenwertlogik) & Mittel (parametergängig, sensorgeometrieabhängig) & Mittel & Gering \\ \hline
      Region-Growing / graphenbasiertes Clustering & Niedrig bis mittel (iterative Nachbarschaftsprüfung) & Mittel bis hoch (reduziert Fehlklassifikationen) & Hoch & Mittel bis hoch \\ \hline
      Range-Image-Methoden & Hoch (2D-Nachbarschaften in Millisekunden) & Mittel (robust gegen Rauschen, sensibel bei Überhängen) & Mittel bis hoch & Gering bis mittel \\ \hline
      Probabilistische Graphmodelle (MRF/CRF) & Niedrig (iterative Inferenz) & Hoch (konsistente Nachbarschaftsmodellierung) & Hoch & Hoch \\ \hline
      Point-/Voxel-basierte Netze (PointNet, PointPillars) & Mittel (GPU-Echtzeit bei optimierter Implementierung) & Hoch (lernbasierte Generalisierung) & Sehr hoch & Hoch \\ \hline
      Bildbasierte CNNs auf Range Images (z.B. RangeNet++) & Hoch (2D-CNN-Inferenz auf GPU) & Mittel bis hoch (robust auf strukturierten Projektionen) & Hoch & Mittel bis hoch \\ \hline
      Spezialisierte Netze (GndNet) & Hoch (Inference $\approx$ 18\,ms/Frame) & Mittel bis hoch (geländeadaptiv trainierbar) & Hoch (\ac{IoU} $\approx$ 84\,\%) & Mittel \\ \hline
    \end{tabular}%
  }
  \caption{Vergleich der in Abschnitt~\ref{chap:bodenmethoden} beschriebenen Bodensegmentierungsmethoden nach Echtzeitfähigkeit, Robustheit, Genauigkeit und Rechenaufwand (nach~\cite{gomes2023survey}).}
  \label{tab:boden-methodenvergleich-ransac}
\end{table}

\section{RANSAC-Verfahren}
\label{sec:ransac_verfahren}

Das \textit{Random Sample Consensus} (\ac{RANSAC}) bietet eine robuste Ebenenabschätzung, indem es Ausreißer konsequent ignoriert und lediglich Punkte berücksichtigt, die innerhalb eines orthogonalen Abstands zur Modellhypothese liegen.
Im Unterschied zu raster- oder morphologiebasierten Ansätzen entsteht so eine präzise Modellierung auch bei schrägen oder partiell verdeckten Fahrbahnen, zu Lasten eines moderaten Iterationsaufwands und einer sorgfältigen Schwellenwertwahl.
Aufgrund dieses Gleichgewichts zwischen Genauigkeit und Implementierungskomplexität dient \ac{RANSAC} in dieser Arbeit als Grundlage der Bodensegmentierung.

Das Verfahren wiederholt folgende Schritte, bis entweder das Iterationsbudget ausgeschöpft ist oder ein Modell die gewünschte Sicherheit erreicht:
\begin{enumerate}
  \item \textbf{Stichprobenbildung:} Aus der Punktwolke werden minimal viele Punkte (\(n=3\) für Ebenen) zufällig gewählt, um einen ersten Ebenenvorschlag zu erzeugen.
  \item \textbf{Modellhypothese:} Aus den Stichprobenpunkten wird eine Ebenengleichung \(a x + b y + c z + d = 0\) berechnet, die Normalenrichtung und Lage der Fahrbahn repräsentiert.
  \item \textbf{Inlier-Prüfung:} Jeder Punkt wird über seinen orthogonalen Abstand
    \[\mathrm{dist}(\mathbf{p}, \Pi) = \frac{|a x + b y + c z + d|}{\sqrt{a^2 + b^2 + c^2}}\]
    zum Ebenenmodell bewertet. Punkte unterhalb des Schwellenwerts \texttt{distance\_threshold} gelten als Inlier.
  \item \textbf{Bewertung und Wiederholung:} Die Hypothese mit den meisten Inliern wird gemerkt; anschließend werden neue Stichproben gezogen, bis die gewünschte Erfolgswahrscheinlichkeit erreicht ist. Aus dem Verhältnis von Inlier-Anteil \(w\) und Maximaliterationen \(k\) ergibt sich \(1-(1-w^n)^k\) als Wahrscheinlichkeit, mindestens eine ausreißerfreie Stichprobe gesehen zu haben.
\end{enumerate}
Die resultierende Ebene trennt Boden- von Hindernispunkten durch ihre senkrechte Distanz. Der iterative Stichprobencharakter geht auf den ursprünglichen \ac{RANSAC}-Vorschlag von Fischler und Bolles zurück, der explizit eine robuste Parameterschätzung bei hohem Ausreißeranteil adressiert \parencite{fischler1981ransac}. Für Punktwolken und Ebenenextraktion wurde das Verfahren unter anderem von Schnabel et~al. für große 3D-Datensätze adaptiert \parencite{Schnabel2007Efficient}. Abbildung~\ref{fig:plane_fitting} zeigt die geometrische Interpretation dieser Distanzklassifizierung nach \textcite{gomes2023survey}.

\begin{figure}[H]
  \centering
  \includegraphics[width=\textwidth]{Bilder/plane_fitting.png}
  \caption{Visuelle Darstellung einer orthogonalen Distanzklassifizierung (~\cite{gomes2023survey})}
  \label{fig:plane_fitting}
\end{figure}

\section{Implementierung}

\label{sec:implementierung_ransac}

Die Bodensegmentierung wurde als eigenständiger \ac{ROS}~2-Knoten in \texttt{C++} mit der \textit{Point Cloud Library (\ac{PCL})} implementiert. 
Der Knoten \texttt{ransac\_ground\_node} abonniert gefilterte \ac{LiDAR}-Punktwolken, schätzt eine Bodenebene mittels \textit{SAC-\ac{RANSAC}} und veröffentlicht eine Hindernis-Punktwolke, aus der die Bodenpunkte entfernt wurden. 
Die Architektur ist strikt streaming-orientiert (Callback-basiert) und verzichtet auf Blockierungen, wodurch eine niedrige Latenz erzielt wird.

Der Knoten deklariert die Ein- und Ausgabetopics als Parameter und nutzt \texttt{SensorDataQoS}:
\begin{itemize}
  \item \textbf{Eingabe (\texttt{input\_topic})}: \texttt{/points\_voxel} \ \emph{(Typ: \texttt{sensor\_msgs/PointCloud2})}, enthält bereits per VoxelGrid ausgedünnte Punkte (\texttt{pcl::PointXYZI}).
  \item \textbf{Ausgabe (\texttt{output\_topic})}: \texttt{/obstacle\_points} \ \emph{(Typ: \texttt{sensor\_msgs/PointCloud2})}, enthält ausschließlich Nicht-Boden-Punkte.
\end{itemize}
Leere Eingaben werden verworfen, um unnötige Rechenarbeit zu vermeiden.

Die wesentlichen Laufzeitparameter werden als \ac{ROS}-Parameter deklariert und können über Launch-Dateien oder \texttt{ros2 param} angepasst werden (Default-Werte aus dem Code):
\begin{table}[h!]
\centering
\begin{tabular}{|l|c|l|}
\hline
\textbf{Parameter} & \textbf{Default} & \textbf{Bedeutung} \\ \hline
\texttt{input\_topic} & \texttt{/points\_voxel} & Eingangs-Punktwolke (vorsegmentiert per VoxelGrid) \\ \hline
\texttt{output\_topic} & \texttt{/obstacle\_points} & Ausgabe ohne Bodenpunkte \\ \hline
\texttt{distance\_threshold} & \SI{0,15}{\meter} & maximaler Punkt-zu-Ebene-Abstand für Inlier \\ \hline
\texttt{max\_iterations} & $1000$ & maximale \ac{RANSAC}-Iterationen \\ \hline
\end{tabular}
\caption{\ac{RANSAC}-Parametrisierung des \texttt{ransac\_ground\_node}.}
\label{tab:ransac_params}
\end{table}

Zur Ebenenschätzung wird \texttt{pcl::SACSegmentation} mit \texttt{SACMODEL\_PLANE} und \texttt{SAC\_RANSAC} verwendet, inkl.\ Koeffizientenoptimierung:
\begin{enumerate}
  \item \textbf{Konvertierung}: \texttt{sensor\_msgs/PointCloud2} $\rightarrow$ \texttt{pcl::PointCloud<pcl::PointXYZI>}.
  \item \textbf{\ac{RANSAC}-Konfiguration}: \texttt{setOptimizeCoefficients(true)}, \texttt{setModelType(PLANE)}, \texttt{setMethodType(\ac{RANSAC})}, \texttt{setDistanceThreshold}, \texttt{setMaxIterations}.
  \item \textbf{Segmentierung}: \texttt{seg.segment(inliers, coefficients)} liefert Inlier-Indizes der Bodenpunkte und Ebenenparameter $\mathbf{n}=(a,b,c)$, $d$ der Ebene
  \[
    a x + b y + c z + d = 0.
  \]
  \item \textbf{Extraktion der Hindernisse}: \texttt{pcl::ExtractIndices} mit \texttt{setNegative(true)} filtert alle Punkte \emph{außerhalb} der Bodenenebene (\,$>$ \texttt{distance\_threshold}\,).
  \item \textbf{Publikation}: Rückkonvertierung nach \texttt{PointCloud2} und Veröffentlichung auf \texttt{/obstacle\_points} (mit ursprünglichem Header/Frame).
\end{enumerate}
Ein Punkt $\mathbf{p}=(x,y,z)^\top$ zählt als Inlier, wenn sein orthogonaler Abstand zur Ebene
\[
  \mathrm{dist}(\mathbf{p}, \Pi) \;=\; \frac{|a x + b y + c z + d|}{\sqrt{a^2+b^2+c^2}}
\]
kleiner gleich \texttt{distance\_threshold} ist.

\section{Ergebnis}
Als Ergebnis ergibt sich eine Punktwolke, die frei von Bodenpunkten ist (siehe Abbildung~\ref{fig:ransac_compare}). 
\begin{figure}[H]
  \centering
  \includegraphics[width=\textwidth]{Bilder/ransac_compare.png}
  \caption{Vergleich der ursprünglichen Punktwolke (links) und der nach dem \ac{RANSAC}-Algorithmus gefilterten Punktwolke (rechts) (2025-10-15-Messdaten/2025-10-15-VK-OL-005)}
  \label{fig:ransac_compare}
\end{figure}

\section{Zusammenfassung}
Das implementierte RANSAC-Verfahren ermöglicht eine robuste und effiziente Trennung von Boden- und Hindernispunkten. Durch die iterative Ebenenschätzung werden auch bei leicht geneigten Fahrbahnen zuverlässige Ergebnisse erzielt, während der Rechenaufwand durch die vorangegangene Voxel-Filterung gering bleibt.

Die resultierende Hindernis-Punktwolke bildet die Eingangsdaten für das folgende Kapitel, in dem zusammengehörige Punkte zu Clustern gruppiert und als Objekte abstrahiert werden.